%%%%%%%%%%%%%%%%%%%%%%%%%%%%%%%%%%%%%%%%%%%%%%%%%%%%
% The following line again needs to be copied
% into Anki:
% -*- coding: utf-8-unix -*-
%%%%%%%%%%%%%%%%%%%%%%%%%%%%%%%%%%%%%%%%%%%%%%%%%%%%
% The first part of the header needs to be copied
%       into the note options in Anki.
%%%%%%%%%%%%%%%%%%%%%%%%%%%%%%%%%%%%%%%%%%%%%%%%%%%%

% layout in Anki:
\documentclass[11pt]{article}
\usepackage[a4paper]{geometry}
\geometry{paperwidth=.5\paperwidth,paperheight=100in,left=2em,right=2em,bottom=1em,top=2em}
\pagestyle{empty}
\setlength{\parindent}{0in}

% encoding:
\usepackage[T1]{fontenc}
\usepackage[utf8]{inputenc}
\usepackage{lmodern}

% packages:
\usepackage{amsmath,amsthm,amsfonts,amssymb}
\usepackage{algorithm}
\usepackage{algpseudocode}
\usepackage{enumerate}

% personal commands:
\newcommand{\dotp}[2]{\langle#1,#2\rangle}
\newcommand{\enscond}[2]{\lbrace#1,#2\rbrace}
\newcommand{\pd}[2]{\frac{\partial#1}{\partial#2} }
\newcommand{\umax}[1]{\underset{#1}{\max}\;}
\newcommand{\umin}[1]{\underset{#1}{\min}\;}
\newcommand{\uargmin}[1]{\underset{#1}{argmin}\;}
\newcommand{\ulim}[2]{\underset{#1\to#2}{\lim}\;}
\newcommand{\dlim}[2]{\underset{#1\to#2}{\longrightarrow}}
\newcommand{\abs}[1]{\left|#1\right|}
\newcommand{\norm}[1]{\left\lVert#1\right\rVert}
\newcommand{\normtwo}[1]{\left\lVert#1\right\rVert_{2}}
\newcommand{\norminf}[1]{\left\lVert#1\right\rVert_{\infty}}
\newcommand{\normf}[1]{\left\lVert#1\right\rVert_{F}}
\newcommand{\ZZ}{\mathbb{Z}}
\newcommand{\NN}{\mathbb{N}}
\newcommand{\CC}{\mathbb{C}}
\newcommand{\RR}{\mathbb{R}}
\newcommand{\EE}{\mathbb{E}}
\newcommand{\PP}{\mathbb{P}}
\newcommand{\Aa}{\mathcal{A}}
\newcommand{\Bb}{\mathcal{B}}
\newcommand{\Ff}{\mathcal{F}}
\newcommand{\Hh}{\mathcal{H}}
\newcommand{\Pp}{\mathcal{P}}
\newcommand{\qandq}{\quad\text{and}\quad}
\newcommand{\qwhereq}{\quad\text{where}\quad}
\newcommand{\qifq}{\quad\text{if}\quad }
\newcommand{\qarrq}{\quad\Longrightarrow\quad }
\newcommand{\pare}[1]{\left(#1\right)}
\newcommand{\brac}[1]{\left[#1\right]}
\newcommand{\function}[5]{\begin{array}[t]{lrcl}
                            #1: & #2 & \longrightarrow & #3 \\
                                & #4 & \longmapsto & #5
                          \end{array}
                          }
\newcommand{\fdomdef}[3]{#1:#2\rightarrow#3}
\newcommand*\dom{\mathop{}\!\mathrm{dom}}

% Specific anki commands
\renewcommand{\cite}[1]{\par\bigskip\hfill{\color{gray}\tiny\(\to\) #1}}
\newcommand{\hide}[1]{\parbox{0cm}{\raisebox{-7pt}[0cm]{\dots}}\color{white}#1\color{black}}
\let\olddots\dots
\renewcommand{\dots}{\,\olddots\,}

%%% Local Variables:
%%% mode: latex
%%% TeX-master: t
%%% End:

% -*- coding: utf-8-unix -*-
%%%%%%%%%%%%%%%%%%%%%%%%%%%%%%%%%%%%%%%%%%%%%%%%%%%%
% Following part of header NOT to be copied into
%            the note options in Anki.
%          ! Anki will throw an errow !
%%%%%%%%%%%%%%%%%%%%%%%%%%%%%%%%%%%%%%%%%%%%%%%%%%%%%
%
%  pdf layout:
%
%  Working original template
\geometry{paperheight=74.25mm}
\usepackage{pgfpages}
\pagestyle{empty}
\pgfpagesuselayout{8 on 1}[a4paper,border shrink=0cm]
\makeatletter
\@tempcnta=1\relax
\loop\ifnum\@tempcnta<9\relax
\pgf@pset{\the\@tempcnta}{bordercode}{\pgfusepath{stroke}}
\advance\@tempcnta by 1\relax
\repeat
\makeatother
%
%  notes, fields, tags:
%
\newcommand{\xfield}[1]{
        #1\par
        \vfill
        {\tiny\texttt{\parbox[t]{\textwidth}{\localtag\\globaltag\hfill\uuid}}}
        \newpage}
\newenvironment{field}{}{\newpage}
\newif\ifnote
\newenvironment{note}{\notetrue}{\notefalse}
\newcommand{\localtag}{}
\newcommand{\globaltag}{}
\newcommand{\uuid}{}
\newcommand{\tags}[1]{
    \ifnote
        \renewcommand{\localtag}{#1}
    \else
        \renewcommand{\globaltag}{#1}
    \fi
    }
\newcommand{\xplain}[1]{\renewcommand{\uuid}{#1}}
%

%%% Local Variables:
%%% mode: latex
%%% TeX-master: t
%%% End:

\begin{document}
%%%%%%%%%%%%%%%%%%%%%%%%%%%%%%%%%%%%%%%%%%%%%%%%%%%%
\tags{test}
\begin{note}
  \xplain{aea0a20e-0633-4827-b49c-d0e3d06ec8ca}
  % Front
  \begin{field}
    Définition d'une variable aléatoire de Bernoulli
  \end{field}
  % Back
  \begin{field}
    On appelle variable aléatoire de Bernoulli de paramètre $p \in
    [0,1]$ une variable aléatoire à valeurs dans $\{0,1\}$ telle que
    $$\PP(X=1)=p\text{ et }\PP(X=0)=1-p$$
  \end{field}
  % Extra
  \begin{field}
  \end{field}
  \xplain{Aléatoire}
  \xplain{Espace fini ou dénombrable}
\end{note}


\begin{note}
  \xplain{cdd6d107-c03b-4cff-88da-2dc06c876b61}
  % Front
  \begin{field}
    L'espérance et la variance d'une variable aléatoire de Bernoulli
    de paramètre p valent respectivement
  \end{field}
  % Back
  \begin{field}
    \begin{itemize}
    \item $\EE(X) = p$
    \item $Var(X) = (1-p)p$
    \end{itemize}

  \end{field}
  % Extra
  \begin{field}

  \end{field}
  \xplain{Aléatoire}
  \xplain{Espace fini ou dénombrable}
\end{note}

\begin{note}
  \xplain{a9349a81-dd51-46f1-9492-7bd8c9eec8ae}
  % Front
  \begin{field}
    Définition d'une variable aléatoire binomiale
  \end{field}
  % Back
  \begin{field}
    On dit que la variable aléatoire $X$ est une variable aléatoire
    binomiale de paramètres $n$ et $p$, que l'on note $\Bb(n,p)$, si X
    prend ses valeurs dans $\{0,\dots,n\}$ et si pour $k \in
    \{0,\dots,n\}$
    $$\PP(X=k)=\binom{n}{k}p^k(1-p)^{n-k}$$
  \end{field}
  % Extra
  \begin{field}

  \end{field}
  \xplain{Aléatoire}
  \xplain{Espace fini ou dénombrable}
\end{note}

\begin{note}
  \xplain{4b47f602-7d78-41a0-b545-86bcef951595}
  % Front
  \begin{field}
    Fonction caractéristique, espérance et variance d'une variable
    aléatoire binomiale
  \end{field}
  % Back
  \begin{field}
    \begin{itemize}
    \item $G_X(s) = (1-p+ps)^n$
    \item $\EE(X) = np$
    \item $Var(X) = np(1-p)$
    \end{itemize}
  \end{field}
  % Extra
  \begin{field}

  \end{field}
  \xplain{Aléatoire}
  \xplain{Espace fini ou dénombrable}
\end{note}

\begin{note}
  \xplain{86df9930-a2bd-42b4-8128-123f3c18c991}
  % Front
  \begin{field}
    Comment peut-on interpréter la loi binomiale?
  \end{field}
  % Back
  \begin{field}
    La loi binomiale modélise une somme d'expérience de Bernoulli.
  \end{field}
  % Extra
  \begin{field}

  \end{field}
  \xplain{Aléatoire}
  \xplain{Espace fini ou dénombrable}
\end{note}

\begin{note}
  \xplain{90bae025-b728-492f-bee5-6a06e230d9e0}
  % Front
  \begin{field}
    Définition d'une variable aléatoire géometrique
  \end{field}
  % Back
  \begin{field}
    On appele variable aléatoire géometrique de paramètre $p
    \in [0,1]$ une variable aléatoire à valeurs dans $\NN^*$ telle
    que $\forall k \in \NN^*$
    $$\PP(X=k)=p(1-p)^{k-1}$$
  \end{field}
  % Extra
  \begin{field}

  \end{field}
  \xplain{}                               % Source
  \xplain{}                               % Section
\end{note}

\begin{note}
  \xplain{254f7d03-9659-4102-abc3-814915e6e59f}
  % Front
  \begin{field}
    Espérance d'une variable aléatoire géometrique
  \end{field}
  % Back
  \begin{field}
    $\EE(X)=\frac{1}{p}$
  \end{field}
  % Extra
  \begin{field}

  \end{field}
  \xplain{}                               % Source
\end{note}


\end{document}

%%% Local Variables:
%%% mode: latex
%%% TeX-master: t
%%% End:
