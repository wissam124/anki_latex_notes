% -*- coding: utf-8-unix -*-
%%%%%%%%%%%%%%%%%%%%%%%%%%%%%%%%%%%%%%%%%%%%%%%%%%%%
% The first part of the header needs to be copied
%       into the note options in Anki.
%%%%%%%%%%%%%%%%%%%%%%%%%%%%%%%%%%%%%%%%%%%%%%%%%%%%

% layout in Anki:
\documentclass[11pt]{article}
\usepackage[a4paper]{geometry}
\geometry{paperwidth=.5\paperwidth,paperheight=100in,left=2em,right=2em,bottom=1em,top=2em}
\pagestyle{empty}
\setlength{\parindent}{0in}

% encoding:
\usepackage[T1]{fontenc}
\usepackage[utf8]{inputenc}
\usepackage{lmodern}

% packages:
\usepackage{amsmath,amsthm,amsfonts,amssymb}
\usepackage{algorithm}
\usepackage{algpseudocode}
\usepackage{enumerate}

% personal commands:
\newcommand{\dotp}[2]{\langle#1,#2\rangle}
\newcommand{\enscond}[2]{\lbrace#1,#2\rbrace}
\newcommand{\pd}[2]{\frac{\partial#1}{\partial#2} }
\newcommand{\umax}[1]{\underset{#1}{\max}\;}
\newcommand{\umin}[1]{\underset{#1}{\min}\;}
\newcommand{\uargmin}[1]{\underset{#1}{argmin}\;}
\newcommand{\ulim}[2]{\underset{#1\to#2}{\lim}\;}
\newcommand{\dlim}[2]{\underset{#1\to#2}{\longrightarrow}}
\newcommand{\abs}[1]{\left|#1\right|}
\newcommand{\norm}[1]{\left\lVert#1\right\rVert}
\newcommand{\normtwo}[1]{\left\lVert#1\right\rVert_{2}}
\newcommand{\norminf}[1]{\left\lVert#1\right\rVert_{\infty}}
\newcommand{\normf}[1]{\left\lVert#1\right\rVert_{F}}
\newcommand{\ZZ}{\mathbb{Z}}
\newcommand{\NN}{\mathbb{N}}
\newcommand{\CC}{\mathbb{C}}
\newcommand{\RR}{\mathbb{R}}
\newcommand{\EE}{\mathbb{E}}
\newcommand{\PP}{\mathbb{P}}
\newcommand{\Aa}{\mathcal{A}}
\newcommand{\Bb}{\mathcal{B}}
\newcommand{\Ff}{\mathcal{F}}
\newcommand{\Hh}{\mathcal{H}}
\newcommand{\Pp}{\mathcal{P}}
\newcommand{\qandq}{\quad\text{and}\quad}
\newcommand{\qwhereq}{\quad\text{where}\quad}
\newcommand{\qifq}{\quad\text{if}\quad }
\newcommand{\qarrq}{\quad\Longrightarrow\quad }
\newcommand{\pare}[1]{\left(#1\right)}
\newcommand{\brac}[1]{\left[#1\right]}
\newcommand{\function}[5]{\begin{array}[t]{lrcl}
                            #1: & #2 & \longrightarrow & #3 \\
                                & #4 & \longmapsto & #5
                          \end{array}
                          }
\newcommand{\fdomdef}[3]{#1:#2\rightarrow#3}
\newcommand*\dom{\mathop{}\!\mathrm{dom}}

% Specific anki commands
\renewcommand{\cite}[1]{\par\bigskip\hfill{\color{gray}\tiny\(\to\) #1}}
\newcommand{\hide}[1]{\parbox{0cm}{\raisebox{-7pt}[0cm]{\dots}}\color{white}#1\color{black}}
\let\olddots\dots
\renewcommand{\dots}{\,\olddots\,}

%%% Local Variables:
%%% mode: latex
%%% TeX-master: t
%%% End:
