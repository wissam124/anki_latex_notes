%%%%%%%%%%%%%%%%%%%%%%%%%%%%%%%%%%%%%%%%%%%%%%%%%%%%
% The following line again needs to be copied
% into Anki:
% -*- coding: utf-8-unix -*-
%%%%%%%%%%%%%%%%%%%%%%%%%%%%%%%%%%%%%%%%%%%%%%%%%%%%
% The first part of the header needs to be copied
%       into the note options in Anki.
%%%%%%%%%%%%%%%%%%%%%%%%%%%%%%%%%%%%%%%%%%%%%%%%%%%%

% layout in Anki:
\documentclass[11pt]{article}
\usepackage[a4paper]{geometry}
\geometry{paperwidth=.5\paperwidth,paperheight=100in,left=2em,right=2em,bottom=1em,top=2em}
\pagestyle{empty}
\setlength{\parindent}{0in}

% encoding:
\usepackage[T1]{fontenc}
\usepackage[utf8]{inputenc}
\usepackage{lmodern}

% packages:
\usepackage{amsmath,amsthm,amsfonts,amssymb}
\usepackage{algorithm}
\usepackage{algpseudocode}
\usepackage{enumerate}

% personal commands:
\newcommand{\dotp}[2]{\langle#1,#2\rangle}
\newcommand{\enscond}[2]{\lbrace#1,#2\rbrace}
\newcommand{\pd}[2]{\frac{\partial#1}{\partial#2} }
\newcommand{\umax}[1]{\underset{#1}{\max}\;}
\newcommand{\umin}[1]{\underset{#1}{\min}\;}
\newcommand{\uargmin}[1]{\underset{#1}{argmin}\;}
\newcommand{\ulim}[2]{\underset{#1\to#2}{\lim}\;}
\newcommand{\dlim}[2]{\underset{#1\to#2}{\longrightarrow}}
\newcommand{\abs}[1]{\left|#1\right|}
\newcommand{\norm}[1]{\left\lVert#1\right\rVert}
\newcommand{\normtwo}[1]{\left\lVert#1\right\rVert_{2}}
\newcommand{\norminf}[1]{\left\lVert#1\right\rVert_{\infty}}
\newcommand{\normf}[1]{\left\lVert#1\right\rVert_{F}}
\newcommand{\ZZ}{\mathbb{Z}}
\newcommand{\NN}{\mathbb{N}}
\newcommand{\CC}{\mathbb{C}}
\newcommand{\RR}{\mathbb{R}}
\newcommand{\EE}{\mathbb{E}}
\newcommand{\PP}{\mathbb{P}}
\newcommand{\Aa}{\mathcal{A}}
\newcommand{\Bb}{\mathcal{B}}
\newcommand{\Ff}{\mathcal{F}}
\newcommand{\Hh}{\mathcal{H}}
\newcommand{\Pp}{\mathcal{P}}
\newcommand{\qandq}{\quad\text{and}\quad}
\newcommand{\qwhereq}{\quad\text{where}\quad}
\newcommand{\qifq}{\quad\text{if}\quad }
\newcommand{\qarrq}{\quad\Longrightarrow\quad }
\newcommand{\pare}[1]{\left(#1\right)}
\newcommand{\brac}[1]{\left[#1\right]}
\newcommand{\function}[5]{\begin{array}[t]{lrcl}
                            #1: & #2 & \longrightarrow & #3 \\
                                & #4 & \longmapsto & #5
                          \end{array}
                          }
\newcommand{\fdomdef}[3]{#1:#2\rightarrow#3}
\newcommand*\dom{\mathop{}\!\mathrm{dom}}

% Specific anki commands
\renewcommand{\cite}[1]{\par\bigskip\hfill{\color{gray}\tiny\(\to\) #1}}
\newcommand{\hide}[1]{\parbox{0cm}{\raisebox{-7pt}[0cm]{\dots}}\color{white}#1\color{black}}
\let\olddots\dots
\renewcommand{\dots}{\,\olddots\,}

%%% Local Variables:
%%% mode: latex
%%% TeX-master: t
%%% End:

% -*- coding: utf-8-unix -*-
%%%%%%%%%%%%%%%%%%%%%%%%%%%%%%%%%%%%%%%%%%%%%%%%%%%%
% Following part of header NOT to be copied into
%            the note options in Anki.
%          ! Anki will throw an errow !
%%%%%%%%%%%%%%%%%%%%%%%%%%%%%%%%%%%%%%%%%%%%%%%%%%%%%
%
%  pdf layout:
%
%  Working original template
\geometry{paperheight=74.25mm}
\usepackage{pgfpages}
\pagestyle{empty}
\pgfpagesuselayout{8 on 1}[a4paper,border shrink=0cm]
\makeatletter
\@tempcnta=1\relax
\loop\ifnum\@tempcnta<9\relax
\pgf@pset{\the\@tempcnta}{bordercode}{\pgfusepath{stroke}}
\advance\@tempcnta by 1\relax
\repeat
\makeatother
%
%  notes, fields, tags:
%
\newcommand{\xfield}[1]{
        #1\par
        \vfill
        {\tiny\texttt{\parbox[t]{\textwidth}{\localtag\\globaltag\hfill\uuid}}}
        \newpage}
\newenvironment{field}{}{\newpage}
\newif\ifnote
\newenvironment{note}{\notetrue}{\notefalse}
\newcommand{\localtag}{}
\newcommand{\globaltag}{}
\newcommand{\uuid}{}
\newcommand{\tags}[1]{
    \ifnote
        \renewcommand{\localtag}{#1}
    \else
        \renewcommand{\globaltag}{#1}
    \fi
    }
\newcommand{\xplain}[1]{\renewcommand{\uuid}{#1}}
%

%%% Local Variables:
%%% mode: latex
%%% TeX-master: t
%%% End:

\begin{document}
%%%%%%%%%%%%%%%%%%%%%%%%%%%%%%%%%%%%%%%%%%%%%%%%%%%%

\tags{optimisation_ML optimisation_non_convexe IASD}
\begin{note}
  \xplain{07079a38-e3a7-4e78-9387-3c6b75ec58ed} % ID
  \begin{field}
    1. Rappeler la définition de la matrice Hessienne pour une fonction
    $f:\RR^n \rightarrow \RR$.\\
    2. Rappeler les conditions nécessaires si $x$ est un minimum local.\\
    3. A-t-on une réciproque?
  \end{field}
  \begin{field}
    1. $$Hess(f)(x) = \pare{\frac{\partial^2 f}{\partial x_i \partial y_j}}_{1
      \leq i \leq j \leq n}$$

    2. $\forall x \in \RR^n$, si $x$ est un minimiseur de $f$, alors
    \begin{itemize}
    \item $\nabla f(x) = 0$
    \item $Hess(f)(x) \succcurlyeq 0$
    \end{itemize}

    3. Si $\nabla f(x)=0$ et $Hess(f)(x) \succ 0$, alors $x$ est un
    minimiseur local
  \end{field}
  \begin{field}
    Contre-example pour la réciproque: considérer $f(x) = x^3$
  \end{field}
  \xplain{Optimisation pour ML}     % Source
  \xplain{Optimisation non-convexe} % Section
\end{note}

\begin{note}
  \xplain{b966699c-1abd-4313-b1b6-89516a7371a9} % ID
  % Front
  \begin{field}
    Soit $x \in \RR^n$ et $f : \RR^n \mapsto \RR$
    \begin{itemize}
    \item $x$ est un point critique de premier ordre si $\dots$
    \item $x$ est un point critique de deuxième ordre si $\dots$
    \end{itemize}
  \end{field}
  % Back
  \begin{field}
    \begin{itemize}
    \item $\nabla f(x) = 0$
    \item $\nabla f(x) = 0$ et $Hess(f)(x) \succcurlyeq 0$
    \end{itemize}
  \end{field}
  % Extra
  \begin{field}

  \end{field}
  \xplain{Optimisation pour ML}     % Source
  \xplain{Optimisation non-convexe} % Section
\end{note}

\begin{note}
  \xplain{490c0cfb-e047-466f-808a-40af5f13ad5a} % ID
  \begin{field}
    Supposons $f$ non convexe, L-lisse et de minimiseur $x_*$.\\
    On effectue $T$ étapes de
    gradient descent avec un pas $\frac{1}{L}$.\\
    On note $T_{min} =
    \uargmin{0\leq t \leq T}\nabla f(x_t)$. \\
    1. Que peut-on dire sur $\norm{\nabla f(x_{T_{min}})}$? \\
    2. Que peut-on dire sur $x_{T_{min}}$?
  \end{field}
  \begin{field}
    $$\norm{\nabla f(x_{T_{min}})} \leq \sqrt{\frac{2L(f(x_0) -
        f(x_*))}{T}}$$
    On dit que $x_{T_{min}}$ est un point critique d'ordre 1 à
    l'approximation $O\pare{\frac{1}{\sqrt{T}}}$
  \end{field}
  \begin{field}
    \underline{Dém}: $f$ est L-lisse donc $f(x_{t+1}) \leq f(x_t) -
    \frac{1}{2L}\norm{\nabla(f(x_{t}))}$
  \end{field}
  \xplain{Optimisation pour ML}     % Source
  \xplain{Optimisation non-convexe} % Section
\end{note}



\begin{note}
  \xplain{186acaa1-3214-4192-ab91-12e0fc7b619d}
  % Front
  \begin{field}
    Énoncer le théorème de convergence vers un point critique de deuxième ordre
    avec Gradient Descent.
  \end{field}
  % Back
  \begin{field}
    On suppose $f$ différentiable et L-lisse et que
    \begin{itemize}
    \item $f$ a un nombre fini de points critiques d'ordre 1.
    \item $\forall M \in \RR,\{x \in \RR^n, f(x) \leq M\}$ est borné.
    \end{itemize}
    On considère la descente de gradient avec un pas constant $\eta \in
    ]0, \frac{1}{L}[$\\
    Pour presque tout $x_0$, la suite $\pare{x_t}_{t \in \NN}$
    converge vers un point critique d'ordre deux.
  \end{field}
  % Extra
  \begin{field}

  \end{field}
  \xplain{Optimisation pour ML}                               % Source
  \xplain{Optimisation non-convexe}                               % Section
\end{note}


% Descente de gradient
\begin{note}
  \xplain{ec972f4e-898a-4ee1-bd9a-69c3dffc8df8}
  % Front
  \begin{field}
    On suppose $f$ différentiable.\\
    Pour tout $x \in \RR^n$ on définit le 'gradient' de $f$ en $x$ par
  \end{field}
  % Back
  \begin{field}
    $$\nabla f(x) = \pare{\pd{f}{x_1}(x),\dots, \pd{f}{x_n}(x)}$$
  \end{field}
  % Extra
  \begin{field}

  \end{field}
  \xplain{Optimisation pour ML}                               % Source
  \xplain{Descente de gradient}                               % Section
\end{note}

\begin{note}
  \xplain{bf4ebcd8-8b0e-4cbf-b339-63d727eea1ab}
  % Front
  \begin{field}
    On considère une forme quadratique:
    $$\function{f}{\RR^n}{\RR}{x}{\frac{1}{2}\dotp{x}{Mx} + \dotp{x}{b}}$$
    % $$f(x) = \frac{1}{2}\dotp{x, Mx} + \dotp{x}{b}$$
    avec $b \in \RR^n$.\\
    1. Quel est le gradient de $f$?\\
    2. En supposant $M$ inversible, comment s'écrit le minimum de $f$?
  \end{field}
  % Back
  \begin{field}
    1. $\nabla f(x) = Mx+b$\\
    2. $x_*=-M^{-1}b$
  \end{field}
  % Extra
  \begin{field}

  \end{field}
  \xplain{Optimisation pour ML}                               % Source
  \xplain{Descente de gradient}                               % Section
\end{note}

\begin{note}
  \xplain{bc542f6f-3ad5-40d5-bb0d-397c39cf3962}
  % Front
  \begin{field}
    Pour une forme quadratique, peut-on écrire directement une
    relation sur $x_{t+1} - x_*$?
  \end{field}
  % Back
  \begin{field}
    $x_{t+1} - x_* = (I - \alpha_tM)(x_t-x_*)$
  \end{field}
  % Extra
  \begin{field}

  \end{field}
  \xplain{Optimisation pour ML}                               % Source
  \xplain{Descente de gradient}                               % Section
\end{note}

\begin{note}
  \xplain{d56d62a3-5d2d-45a1-bff8-5b1dc9cf5069}
  % Front
  \begin{field}
    Dans l'algorithme de descente de gradient, que risque-t-on si le
    pas $\alpha_t$ est trop grand?
  \end{field}
  % Back
  \begin{field}
    L'approximation linéaire n'est plus valable et on risque de voir
    $f(x_t)_{t\in\NN}$ diverger.
  \end{field}
  % Extra
  \begin{field}

  \end{field}
  \xplain{Optimisation pour ML}                               % Source
  \xplain{Descente de gradient}                               % Section
\end{note}

\begin{note}
  \xplain{6b70dc91-495a-43b2-a636-439c5f52246f}
  % Front
  \begin{field}
    Quel est le principe du Exact Line Search pour choisir le pas pour
    GD?
  \end{field}
  % Back
  \begin{field}
    À chaque étape t, on choisit $\alpha_t = \uargmin{\alpha}(f\pare{x_t -
    \alpha\nabla f(x_t)})$
    \\
    i.e. on choisit le pas optimal tel que à $x_{t+1}$, la valeur de
    la fonction $f$ soit minimale.
  \end{field}
  % Extra
  \begin{field}

  \end{field}
  \xplain{Optimisation pour ML}                               % Source
  \xplain{Descente de gradient}                               % Section
\end{note}

\begin{note}
  \xplain{40fc0bf4-5c8b-481e-b0d6-c7ac3ce6993d}
  % Front
  \begin{field}
    Si $f$ est L-lisse, on a $f(y) \leq f(x) + \dotp{\nabla f(x)}{y-x}
      + \dots$
  \end{field}
  % Back
  \begin{field}
    $f(y) \leq f(x) + \dotp{\nabla f(x)}{y-x}
      + \frac{L}{2}\norm{x-y}^2$
  \end{field}
  % Extra
  \begin{field}
    On pose $\function{\phi}{[0,1]}{\RR}{s}{f((1-s)x+sy)}$ et on écrit
    $f(y)-f(x) = \phi(1) - \phi(0)$
  \end{field}
  \xplain{Optimisation pour ML}                               % Source
  \xplain{Descente de gradient}                               % Section
\end{note}

\begin{note}
  \xplain{b03cf739-0340-4805-9b55-a8f3551c900b}
  % Front
  \begin{field}
    Définition fonction L-lisse
  \end{field}
  % Back
  \begin{field}
    On dit que $f$ est L-lisse si $$\forall x,y \in \RR^n, \norm{\nabla
    f(x) - \nabla f(y)} \leq L\norm{x-y}$$
  \end{field}
  % Extra
  \begin{field}

  \end{field}
  \xplain{Optimisation pour ML}                               % Source
  \xplain{Descente de gradient}                               % Section
\end{note}

\begin{note}
  \xplain{bd004517-38dc-4854-81dc-cb99d8bd20e6}
  % Front
  \begin{field}
    Si $f$ est L-lisse, que peut-on dire sur la suite $\nabla f(x_t)$
    (avec $x_t$ suite de GD avec un pas constant $\alpha_t=\frac{1}{L}$)?\\
    Peut-on dire qqch sur la suite $f(x_t)$?
  \end{field}
  % Back
  \begin{field}
    $\ulim{t}{\infty}\norm{\nabla f(x_t)} = 0$ i.e $(x_t)_{t\in\NN}$ converge vers un
    point critique d'ordre 1.\\
    On ne peut en revanche pas dire à priori que $\ulim{t}{\infty}f(x_t)=f(x_*)$.

  \end{field}
  % Extra
  \begin{field}

  \end{field}
  \xplain{Optimisation pour ML}                               % Source
  \xplain{Descente de gradient}                               % Section
\end{note}

\begin{note}
  \xplain{f5cf86b3-d17b-4699-b5c7-8742f02abeb5}
  % Front
  \begin{field}
    Une forme quadratique de matrice symétrique $M$ est convexe ssi ...
  \end{field}
  % Back
  \begin{field}
    $M$ est semi-définie positive.
  \end{field}
  % Extra
  \begin{field}
    $f(y) = f(x) + \dotp{\nabla f(x)}{y-x} + \frac{1}{2}\dotp{y-x}{M(y-x)}$
  \end{field}
  \xplain{Optimisation pour ML}                               % Source
  \xplain{Descente de gradient}                               % Section
\end{note}


\begin{note}
  \xplain{d772b713-8988-49bf-b4a1-f7860dc1535c}
  % Front
  \begin{field}
    Soit $f$ convexe et L-lisse.\\
    On considère la descente de gradient avec un pas constant $\alpha_t
    = \frac{1}{L}$.\\
    Que peut-on dire sur $f(x_t) - f(x_*)$ pour tout $t \in \NN$?
  \end{field}
  % Back
  \begin{field}
    $$0\leq f(x_t) - f(x_*) \leq \frac{2L}{t+4}\norm{x_0-x_*}^2$$
  \end{field}
  % Extra
  \begin{field}
    \begin{itemize}
    \item Mq $\norm{x_{t+1}-x_*}^2 \leq \norm{x_t-x_*}^2$
    \item Établir une inégalité de récurrence sur $f(x_t) - f(x_*)$
    \end{itemize}
  \end{field}
  \xplain{Optimisation pour ML}                               % Source
  \xplain{Descente de gradient}                               % Section
\end{note}

\begin{note}
  \xplain{e30a6b7d-e798-4f42-9d7b-51f1bafc6241}
  % Front
  \begin{field}
    Si $f$ est différentiable, alors elle est convexe ssi ...
  \end{field}
  % Back
  \begin{field}
    $$\forall x,y \in \RR^n, f(y) \geq f(x) + \dotp{\nabla
      f(x)}{y-x}$$
  \end{field}
  % Extra
  \begin{field}
  \end{field}
  \xplain{Optimisation pour ML}                               % Source
  \xplain{Descente de gradient}                               % Section
\end{note}

\begin{note}
  \xplain{6b65a613-094b-4d37-b663-168675c1f610}
  % Front
  \begin{field}
    Soit $f$ convexe et L-lisse.\\
    Que peut-on dire sur la vitesse de convergence de $f(x_t)$ (algo
    GD)?\\
    Que peut-on dire sur la vitesse de convergence de $\ln(f(x_t) - f(x_*))$?
  \end{field}
  % Back
  \begin{field}
    $f(x_t)$ converge vers $f(x_*)$ avec une vitesse en
    $\mathcal{O}(\frac{1}{t})$.\\
    $\ln(f(x_t) - f(x_*))$ converge vers 0 linéairement en $\ln t$.

  \end{field}
  % Extra
  \begin{field}

  \end{field}
  \xplain{Optimisation pour ML}                               % Source
  \xplain{Descente de gradient}                               % Section
\end{note}

\begin{note}
  \xplain{7198a086-5372-4b1d-afa5-ae776da43419}
  % Front
  \begin{field}
    Définition fonction $\mu$-fortement convexe
  \end{field}
  % Back
  \begin{field}
    Soit $f$ une fonction différentiable et $\mu > 0$.\\
    On dit que $f$ est $\mu$-fortement convexe ssi:
    $$\forall x,y \in \RR^n, f(y) \geq f(x) + \dotp{\nabla f(x)}{y-x}
    + \frac{\mu}{2}\norm{y-x}^2$$
  \end{field}
  % Extra
  \begin{field}

  \end{field}
  \xplain{Optimisation pour ML}                               % Source
  \xplain{Descente de gradient}                               % Section
\end{note}


\begin{note}
  \xplain{5051fcbb-e8a1-4fd9-b262-7dcef1937f04}
  % Front
  \begin{field}
    Pour tout $\mu > 0$, $f$ est $\mu$-fortement convexe ssi $\dots$
  \end{field}
  % Back
  \begin{field}
    $x \mapsto f(x) - \frac{\mu}{2}\norm{x}^2$ est convexe.
  \end{field}
  % Extra
  \begin{field}
    Utiliser la propriété permettant d'obtenir un minorant d'une
    fonction convexe en tout point.
  \end{field}
  \xplain{Optimisation pour ML}                               % Source
  \xplain{Descente de gradient}                               % Section
\end{note}

\begin{note}
  \xplain{1225568d-0961-48a6-9cac-c05d6314e7ad}
  % Front
  \begin{field}
    Théorème de convergence de GD pour une fonction L-lisse et
    $\mu$-fortement convexe.
  \end{field}
  % Back
  \begin{field}
    $\forall t \in \NN, f(x_t)-f(x_*)\leq
    \frac{L}{2}\pare{1-\frac{\mu}{L}}^t\norm{x_0-x_*}^2$
  \end{field}
  % Extra
  \begin{field}
    1. Montrer que $\norm{x_{t+1}-x_*}^2 \leq
    \pare{1-\frac{\mu}{L}}\norm{x_t-x_*}^2$\\
    2. Utiliser propriété L-lisse et $\nabla f(x_*)=0$
  \end{field}
  \xplain{Optimisation pour ML}                               % Source
  \xplain{Descente de gradient}                               % Section
\end{note}

\begin{note}
  \xplain{cd638d46-9422-4daa-8a45-e8310ccfe757}
  % Front
  \begin{field}
    Soit $f$ une fonction $\mu$-fortement convexe et
    L-lisse. Qu'appelle-t-on le nombre de conditionnement?
  \end{field}
  % Back
  \begin{field}
    $K = \frac{L}{\mu} \geq 1$
  \end{field}
  % Extra
  \begin{field}

  \end{field}
  \xplain{Optimisation pour ML}                               % Source
  \xplain{Descente de gradient}                               % Section
\end{note}

\begin{note}
  \xplain{aca8d44d-6270-4212-bea6-730fffc0cba5}
  % Front
  \begin{field}
    On suppose $f$ une forme quadratique de matrice symétrique $M$.
    Quand avons-nous $f$ $\mu$-fortement convexe?
  \end{field}
  % Back
  \begin{field}
    En supposant que $\lambda_1,\dots,\lambda_n$ sont les valeurs
    propres ordonnées de $M$, $f$ est $\lambda_n$-fortement convexe ssi $\lambda_n>0$.
  \end{field}
  % Extra
  \begin{field}

  \end{field}
  \xplain{Optimisation pour ML}                               % Source
  \xplain{Descente de gradient}                               % Section
\end{note}


\tags{optimisation_ML analyse_convexe IASD}
\begin{note}
  \xplain{3ade79c8-6c47-4d2f-b668-b3bac3436552}
  % Front
  \begin{field}
    Soit $f:\RR^N \mapsto \RR \cup \{+\infty\}$, coercive, sci, et
    $f\not\equiv +\infty$.\\
    $f$ admet-elle un minimum sur $\RR^N$?
  \end{field}
  % Back
  \begin{field}
    $f$ admet un minimum sur $\RR^N$ et $\min f < \infty$
  \end{field}
  % Extra
  \begin{field}
    Restreindre $f$ à un compact.
  \end{field}
  \xplain{Optimisation pour ML}                               % Source
  \xplain{Analyse convexe}                               % Section
\end{note}

\begin{note}
  \xplain{540524d1-a931-4506-816d-954f7a270d38}
  % Front
  \begin{field}
    Énoncer le théorème de caractérisation de fonctions convexes, sci,
    propre comme enveloppe supérieure de fonctions affines.
  \end{field}
  % Back
  \begin{field}
    Soit $\fdomdef{f}{\RR^N}{\RR\cup\{+\infty\}}$ convexe, sci, propre.\\
    Alors $\forall x \in \RR^N, f(x)=\sup\{l(x)|l \text{ fonction
      affine tq } l\leq f\}$
  \end{field}
  % Extra
  \begin{field}

  \end{field}
  \xplain{Optimisation pour ML}                               % Source
  \xplain{Analyse convexe}                               % Section
\end{note}

\begin{note}
  \xplain{21aa30a5-bb83-42c7-8b6f-4bb9042c3313}
  % Front
  \begin{field}
    Définition d'une fontion propre
  \end{field}
  % Back
  \begin{field}
    On dit que $\fdomdef{f}{\RR^N}{\RR\cup\{\pm\infty\}}$ est propre si:
    \begin{itemize}
      \item $\dom{f} = \{x\in\RR^N|f(x)<+\infty\} \neq \emptyset$
      \item $\forall x \in \RR^N, f(x) > - \infty$
    \end{itemize}

  \end{field}
  % Extra
  \begin{field}

  \end{field}
  \xplain{Optimisation pour ML}                               % Source
  \xplain{Analyse convexe}                               % Section
\end{note}

\begin{note}
  \xplain{20afdb6f-972e-49e2-a9a7-104bb0e06a2a}
  % Front
  \begin{field}
    Théorème de projection d'un point sur un convexe.
  \end{field}
  % Back
  \begin{field}
    Soit $C \subset \RR^N$ un ensemble convexe, fermé et non vide.\\
    Pour tout $x \in \RR^N$, il existe un unique élément $p \in C$ tel
    que $$\norm{x-p} = \umin{c \in C}\norm{x-c} = d(x,C)$$
    De plus, $p$ est l'unique élément de C tel que $\dotp{x-p}{c-p}
    \leq 0$ pour tout $c \in C$
  \end{field}
  % Extra
  \begin{field}
    Construire une fonction coercive, sci, et $\not\equiv +\infty$.
    Vérifier l'unicité du minimiseur par l'absurde.
  \end{field}
  \xplain{Optimisation pour ML}                               % Source
  \xplain{Analyse convexe}                               % Section
\end{note}

\begin{note}
  \xplain{aba6ed57-921e-47a8-886c-4d97550faa8d}
  % Front
  \begin{field}
    Définition sous-gradient
  \end{field}
  % Back
  \begin{field}
    Soit $f$ une fonction convexe, propre et $p \in \RR^N$.\\
    On dit que $p$ est un sous-gradient de $f$ en $x$ si:
    $$\forall y \in \RR^N, f(y) \geq f(x) + \dotp{p}{y-x}$$
    On le note $p \in \partial{f}(x)$.
  \end{field}
  % Extra
  \begin{field}

  \end{field}
  \xplain{Optimisation pour ML}                               % Source
  \xplain{Analyse convexe}                               % Section
\end{note}

\begin{note}
  \xplain{838539c8-8578-4fa3-80a6-e7c339ebc294}
  % Front
  \begin{field}
    La fonction indicatrice $\chi_A$ est sci ssi ...
  \end{field}
  % Back
  \begin{field}
    A est fermé.
  \end{field}
  % Extra
  \begin{field}

  \end{field}
  \xplain{Optimisation pour ML}                               % Source
  \xplain{Analyse convexe}                               % Section
\end{note}

\begin{note}
  \xplain{37320fdd-c891-4973-bf3f-6a30a79900a6}
  % Front
  \begin{field}
    Que peut-on dire sur l'ensemble de niveau inférieur d'une fonction
    convexe?\\
    La réciproque est-elle vraie?
  \end{field}
  % Back
  \begin{field}
    C'est un ensemble convexe.\\
    Non, considérer par exemple $x \mapsto \sqrt{\norm{x}
    }$
  \end{field}
  % Extra
  \begin{field}

  \end{field}
  \xplain{Optimisation pour ML}                               % Source
  \xplain{Analyse convexe}                               % Section
\end{note}

\end{document}

%%% Local Variables:
%%% mode: latex
%%% TeX-master: t
%%% End:
